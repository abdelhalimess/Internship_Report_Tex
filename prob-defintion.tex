\chapter{Objective}
\section{Problem definition}

The problem addressed during the internship at the National Social Security Fund (CNAS) revolves around optimizing the efficiency and effectiveness of social security processes and services. CNAS serves a vast number of beneficiaries, and it is crucial to ensure that the delivery of social security benefits is streamlined, transparent, and accessible to all eligible individuals.

One of the primary challenges observed during the internship is the need to enhance the digital infrastructure and modernize the existing systems and processes at CNAS. This includes developing user-friendly web-based applications, improving data management and analysis, and implementing advanced technologies to automate and expedite various administrative and operational tasks.

Additionally, there is a growing demand for optimizing the coordination and collaboration between different departments within CNAS. Efficient communication, data sharing, and interdepartmental workflows are vital to ensure seamless service delivery and timely decision-making.

The problem definition also encompasses the need for continuous monitoring, evaluation, and improvement of social security programs and policies. CNAS must remain adaptive and responsive to evolving societal needs and challenges, while also ensuring fiscal sustainability and compliance with legal and regulatory frameworks.

Throughout the internship, efforts were focused on addressing these challenges and proposing practical solutions to enhance the overall performance of CNAS in delivering social security benefits and services. The problem definition served as a guiding framework for the assigned tasks and projects, fostering innovation, efficiency, and effectiveness in the operations of CNAS.

By identifying and addressing these challenges, CNAS can further strengthen its role as a leading institution in social security and provide enhanced support to beneficiaries, contributing to the social and economic development of [Country].

