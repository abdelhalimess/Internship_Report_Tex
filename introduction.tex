\addcontentsline{toc}{chapter}{Introduction}
\chapter*{Introduction}
The following report presents an overview of our internship experience at the National Social Security Fund (CNAS). This internship was conducted as part of my academic program at Hassiba Benbouali University, Ouled Fares, Chlef. The report aims to provide a comprehensive account of the tasks, projects, and learning outcomes achieved during the internship period.

The CNAS is a prominent institution responsible for managing and administering social security programs in Algeria. It plays a vital role in ensuring the welfare and well-being of citizens by providing various social security benefits and services. The internship at CNAS offered an invaluable opportunity to gain practical knowledge and insights into the functioning of such a significant organization.

This report begins by outlining the problem statement and objectives of the internship, providing a clear understanding of the context and scope of the assigned tasks. It then proceeds to describe the work conducted during the internship period, highlighting the key projects, responsibilities, and challenges encountered. Additionally, the report discusses potential future work and recommendations based on the observations made during the internship.

Throughout the internship, we had the opportunity to work with a team of experienced professionals in different departments of CNAS, including the IT department. This collaborative environment allowed us to acquire practical skills, enhance our knowledge of social security systems, and gain insight into the administrative and operational processes of CNAS.

By documenting our experiences and reflections in this report, we aim to share the valuable lessons learned, challenges faced, and accomplishments achieved during our internship at CNAS. It is our hope that this report will serve as a useful reference for individuals interested in the operations and functions of CNAS and provide insights for future interns or researchers in the field of social security.

Please note that certain details and confidential information related to CNAS operations may be omitted or anonymized to ensure the confidentiality and privacy of the organization.

